\section{Related Work}

% What gets filled in by a \cite{} command comes from the bibliography file.

%What other projects, papers, ideas are influential to your project~\cite{cudabook, qt5:web}?  These
%could be academic papers, books, open source code, or anything else that's out there.

We used the particle fountain in-class example as our starter code.  We also incorporated the cube in-class example as the foundation for our static environment. \cite{professor}

In order to improve runtime and help implement collision detection, an octree implementation found on github \cite{octree} and \cite{octree2} was incorperated into our design.

We reviewed the following various research papers and open source libraries to find
informaiton about particle movement, collision interaction, water molecule physics, and
particle simulation animation technieques. To learn about the math behind how water
particles behave we looked at two research papers and borowed technieques from both "Real-time Fluid Simulation Using Height Fields" \cite{fluidSim} and "Particle-based Viscoelastic Fluid Simulation"  \cite{viscoelastic}. We also looked at a google library called LiquidFun which discribes some 
techniques they used to model water pressure and viscosity \cite{liquidfun}.
