\section{Introduction}

Our motivations for this project include the desire to model a non-linear physical system,
implement a dynamic particle animation in real time, and extend some of the topics we covered in class.
While we discussed particle systems in class, we never had the opportunity to work with one extensively,
so we saw this project as a means of exploring this topic in greater depth. Furthermore, we wanted to gain experience working with particle interactions.

Our primary goal in this project is to develop a realistic simulation of water particle movement.
We have developed and implemented functionality that handles collision detection so that water
particles can interact with each other and the environment. Our original goals were as follows.  First, we hoped to implement a particle system that had fast inter-particle and particle-wall collision interaction. After we achieved this, we aimed to work on the physics behind how water particles might interact in such a system. We then hoped to model the behavior of water, and our final goal (and quite a reach) was to abstract away the particles by applying rending techniques to blend the particles together.\\ 

\noindent{\textbf{Timeline of Short Term Goals:}}
\begin{itemize}
  \item Nov 12 - Nov 19: Hand in project proposal and obtain started code from the particle fountain example done in class.
  \item Nov 20 - Nov 26: Implement particles interacting with scene and possibly each other for
  a few particles.
  \item Nov 27 - Dec 3: Start working toward adding more particles while keeping run time reasonable.
  \item Dec 4 - Dec 10: Finish having many particles interact with each other. Start thinking about
  how to model water. Implement more advanced models of water and start figuring out how to make
  the water look more aqueous.
  \item Dec 10 - Dec 19: Finish up report and continue to improve on simulation.
\end{itemize}

Some challenges we expected to face included handling a large amount of particles while maintaining
efficient run-time, handling particle-wall and particle-particle collisions, efficiencly maintaing attributes of individual particles, and difficulty applying complex physics and mathematical equations to better approximate the movement of water. 

Particle systems are often used to depict fluid motion or other chaotic behavior because complicated non-linear dynamics can be simplified to individual particle interactions. Systems like smoke and water can be simulated using other rendering techniques; however, this is often much more complicated. We were interested in using the knowlege we already had about graphics to implement a somewhat realistic physical system.  Finally, this project gave us the opportunity to create a relatively simple and efficient way to model a real world particle system that is relevant and understandable to people with some amount of graphics knowledge.  



