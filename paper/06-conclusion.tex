\section{Conclusion}

%\emph{Briefly} summarize what you did.

%What do you know now that you didn't at the time you started this project?
%What does your work tell us? What could we do next given the current status of the project?

In this project we successfully modeled a particle system which begins to approximate the
movement of water.  The particle system allows for particle-particle collisions as well 
as particle-wall collisions with the environment we created.  The use of an octree
allowed for large performance benefits and the ability to introduce a much larger number 
of particles into the system.  Finally, the particles were made to respond correctly to 
transformations applied to the static environment.

This project greatly expanded our knowlege in the areas of both graphics and physical 
systems in general. First of all, we gained a lot of knowedge about vertex and fragment
shaders.  We learned how to send information to these shaders as well as get information
out of them and used them in more complex ways than we had in previous projects.  We also
obtained a lot of knowlege about octrees, their structure, benefits, and implementation
.  Although we imported the octree implementation from github, we had to incorporate it
into our design and change the implementation to better work with our particle collisions
. The properties of water that were researched also gave us some understanding of the
physics involved in this type of physical system.  Finally, we have a solid understanding
of particle systems.  We have learned about their basic features such as intialization
and drawing as well as more complex ideas such as particles interacting with each other
and with the environment.  By efficiently applying changes to single particles based on
collisions we were able to model a much larger system.

Given the current status of the project, some next steps include:
\begin{itemize}
  \item{Changing the color of particles depending on their depth and collisions}
  \item{Applying more equations that model water-like movement (i.e. surface water movement,
    waves, etc.)- particularly look into the Navier-Stokes and Euler equations of fluid dynamics}
  \item{Making the particles react more fluidly to changes in their environment}
  \item{Further improving runtime with an increased number of particles}
  \item{Apply advanced rendering techniques to blend the individual particles together in order to make the visual output more similar to water}
\end{itemize}

