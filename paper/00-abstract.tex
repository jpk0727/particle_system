\begin{abstract}

\indent{\indent{ 
Our Computer Graphics Final Project is an implementation of a particle system
that attempts to model water. In essence, we have many small particles
that are interacting dynamically with each other, the environment, and are under
the influence of gravity. Our environment consists of a cube with transparent sides.
Our graphic can be rotated in two different
manners: camera rotation and the rotation of the cube.
When the cube is rotated, (in the x, y, or z direction),the particles reflect this 
change and continue to act as they would under normal gravity by continuing to fall downward.
The camera rotation simply acts as though a camera were rotating around the entire
system. In order to achieve particle interaction using a large number of particles without 
sacrificing real time performance, we utilized an octree to help with particle to
particle collision detection. The octree subdivides the cube and checks for collisions only between particles within the same region. This reduces our collision detection run time significantly. Unfortunately, the physics that determines our
particles interaction is more representative of small bouncing balls.
While we began thinking about and implementing more complex particle 
interactions, like water pressure and viscosity, further modifications
to this project could be improving these features.
Our project was successful
in that we implemented a large particle system with fast run time and
complicated particle interactions.}} 



\end{abstract}
