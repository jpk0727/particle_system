\section{Evaluation}

%Did you achieve success?  What metrics should we care about and why?  How did
%you evaluate using those metrics, and what were the results?

This project was an overall success.

Our final output consists of two different scenes.  The first scene begins with a
group of one thousand particles in the formation of a cube dropping into the tank. 
(see Figure 1) 
The particles then respond accordingly to particle-particle collisions and wall-particle
collisions.  If the 'r' key is pressed, one thousand more particles enter the scene
from another dropped cube.  The second scene shows fewer but much larger particles 
coming from a single source at a certain time displacement.  This second scene makes
the accuracy of the particle's behavior clearer to the viewer.  In either scene the
camera can be rotated using the 'x', 'y', and 'z' keys and the cube can be rotated 
using the 'a', 's', and 'd' keys.  These various scenes and interactions provide 
evidence of a high level of understanding of different shaders, writing to VBOs,
and particle interactions.

The metrics used to evaluate our success include the following:
\begin{itemize}
  \item{A simulation of a particle system that can run in real-time}
  \item{The ability to handle a large number of particles without sacrificing runtime}
  \item{Accurate particle interations and collisions}
  \item{Having a particle system interact with both a static and changing environment}
  \item{Increased understanding of particle systems, octrees, and collision detection}
  \item{Accurately modeling the movement of water}
\end{itemize}

A large measure of success was the ability to work with hundreds of particles in realtime.
Because of the implementation of the octree described above, instead of having to check 
every particle against every other particle to determine collisions, any given particle 
had to be check against less than ten other particles.  Specifically, our octree was
implemented so that it could subdivide up to twenty (however depth 5 seems optimal) 
and a maximum of five particles
can be stored in every node (octant).  While checking every particle against every 
other particle would take $bigO$ from $O(n^2)$ with linear particle collision detection to
much less with octree collision detection.  

However, there are still clear limitations to the number of particles that can interact
without seeing the performance suffer.  Our current implementation can handle approximately 
two thousand particles before it begins to get sluggish and get hung up.  Given more time
, it may be beneficial to try either different implementations of an octree or a different
data structure altogether in order to obtain maximum performance.

Another clear measure of success was the accuracy of the particle and wall collisions. 
(The "large" mode of our final project shows these interactions most clearly.) These
collisions model correct real-world physical behavior.  In addition, the particles reacting
to the rotating cube provide further evidence of a robust implementation of particle
interations with the environment.

While originally, we were hoping to implement many more features of water dynamics, this
goal proved to be too much given the time constraint.  However, we managed to do some 
research on the properties of both pressure and viscosity of water and although they are 
not fully or correctly functioning, we have applied them to our particle interations. 
Although our goal of acurate water movement was not fully realized, the research done 
and knowledge gained is a sign of success.  It is also a measure of our success that these 
more acurate models of water can easily be incorporated into our design by simply adding
methods to the particle class or changing the particle collision equation found in the
particle-particle collision method.  

Thus, while more work and reasearch could be done, this project provides a solid, robust,
and well designed basis on which to build a more complex water simulation.  





